%%%%%%%%%%%%%%%%%%%%%%%%%%%%%%%%%%%%%%%%%%%%%%%%%%%%%%%%%%%%%%%%%%%%%%%%%%%
%
% Generic template for TFC/TFM/TFG/Tesis
%
% $Id: agradecimientos.tex,v 1.7 2015/06/05 00:10:31 macias Exp $
%
% By:
%  + Javier Macías-Guarasa. 
%    Departamento de Electrónica
%    Universidad de Alcalá
%  + Roberto Barra-Chicote. 
%    Departamento de Ingeniería Electrónica
%    Universidad Politécnica de Madrid   
% 
% Based on original sources by Roberto Barra, Manuel Ocaña, Jesús Nuevo,
% Pedro Revenga, Fernando Herránz and Noelia Hernández. Thanks a lot to
% all of them, and to the many anonymous contributors found (thanks to
% google) that provided help in setting all this up.
%
% See also the additionalContributors.txt file to check the name of
% additional contributors to this work.
%
% If you think you can add pieces of relevant/useful examples,
% improvements, please contact us at (macias@depeca.uah.es)
%
% You can freely use this template and please contribute with
% comments or suggestions!!!
%
%%%%%%%%%%%%%%%%%%%%%%%%%%%%%%%%%%%%%%%%%%%%%%%%%%%%%%%%%%%%%%%%%%%%%%%%%%%

% Use this if you don't like the fancy style
\thispagestyle{empty}

\chapter*{Agradecimientos}
\label{ch:agradecimientos}
\markboth{Agradecimientos}{Agradecimientos}
\addcontentsline{toc}{chapter}{Agradecimientos}

Tras cuatro años, ha llegado el momento.
Y no me siento ni cuatro años más viejo ni cuatro años más sabio.
De hecho, creo que me siento cuatro años más joven, pues solo me he dado cuenta de lo poco que sé.
Cuatro años de tesis doctoral pueden ser duros, pues es un camino muy abierto para el que nadie tiene la solución: tú eres quien debe encontrarla.
En aquellos momentos en los que me sentía perdido pensaba en aquella frase de Machado: ``\textit{Caminante, no hay camino, se hace camino al andar}''.
Y caminando y caminando, hasta aquí he llegado.

En primer lugar, quien más se merece mis agradecimientos es mi tutor, Roberto.
Fue él quien confió en mí para concederme una de las codiciadas becas de investigación FPI y me acompañó durante todo el proceso.
Gracias a él esta tesis se ha hecho realidad.
Múltiples conversaciones en su despacho, el laboratorio, la cafetería y frente al ordenador han moldeado mi forma de pensar y de entender la investigación.
Y no solo eso, sino que siempre que sus obligaciones se lo han permitido ha sido capaz de aguantar con nosotros hasta altas horas de la madrugada en algún bar perdido de Alcalá.
Espero que sepa que este no es el final, sino que nuestro camino no ha hecho más que comenzar.

También quiero mencionar a todos aquellos que han estado conmigo durante estos años, tanto dentro como fuera del laboratorio.
A todas esas amistades que no solo se han limitado al trabajo y han trascendido al día a día.
A Iván, Rafa ``Rafiki'', Diego ``Folliki'', Nadia, Pablo, Walfrido ``El Meme'', DongXu y Keshi ``DongXuLina'': gracias por hacer mucho más entretenidas las horas en el laboratorio (y en la cafetería) y los pádel ocasionales.

Tampoco quiero olvidarme de los amigos que hice en mi estancia en Tokyo y que me ayudaron a llevar mejor la distancia: Emile, Jessie y Sebastian.
Aún echo de menos las sesiones en el club de rock tocando toe.
Y tampoco puedo olvidarme de Arantza, quien hizo un gran esfuerzo para venir a visitarme y que, a día de hoy, sigue acompañándome.

Aunque no sea común, también quiero agradecer a todas las personas que, con su trabajo y esfuerzo, contribuyen día a día a que existan becas como la que he disfrutado.
En un momento de crispación y crisis social como el que vivimos, es importante reconocer la labor de todos aquellos que hacen posible que jóvenes como yo podamos desarrollar nuestro talento y crecer como personas.
Sé que nuestra sociedad no es perfecta, pero es necesario reconocer también los logros que conseguimos cuando trabajamos todos juntos.
A pesar de la precariedad que sufrimos los jóvenes investigadores, estoy agradecido de haber podido desarrollar mi investigación sin que nadie me haya pedido nada a cambio.

Finalmente, y no por ello menos importante, quiero agradecer a mi familia su apoyo incondicional durante estos años.
Gracias a ellos soy quien soy hoy en día.
Son ellos quienes siempre me han brindado la oportunidad de crecer y aprender para llegar a este momento.
Si no hubiera podido volver a vivir en casa de mis padres, esta tesis no habría tenido lugar, y por ello creo que merecen un agradecimiento especial.

Gracias a todos.

% Back to normal JIC. Use it if you set \pagestyle{myplain} above
%\pagestyle{fancy}

%%% Local Variables:
%%% TeX-master: "../book"
%%% End:


