\textbf{Visual Semantic Navigation.} We can identify in the literature the following groups of works for the VSN problem, depending on the learning paradigm.
In the first group, there are those works that focus on the task of navigating to an object in realistic indoor environments, \eg~\cite{zhu2017, wijmans2020, chang2020, khandelwal2022}, using simulators and an agent based on CNNs as visual encoders and RNNs as the actor-critic head, following a RL paradigm.
The second group consists of the works that address the VSN problem using imitation learning~\cite{wu2020a, ramrakhya2022} to build navigation policies from expert demonstrations.
Finally, in the third set we have the approaches using meta-learning techniques in order to be able to quickly adapt to new environments~\cite{wang2017, wortsman2019, zhang2022}.

Our work belongs to the first group.
In fact, our proposal is a simplification of the approach in~\cite{khandelwal2022}, where we build a model based on a CLIP feature extractor and two LSTMs encoders for the agent state, introducing also reward shaping~\cite{wijmans2020} and $\epsilon\text{-}greedy$~\cite{mnih2013}.

\textbf{Sparsity and exploration methods.} To address the sparse reward and exploration problems, different approaches have been proposed.
Auxiliary tasks~\cite{jaderberg2016, ye2021} help the agent to explore the environment and gather extrinsic reward by maximizing pseudo-reward functions.
Curiosity-driven exploration~\cite{pathak2017} leverages on the error of the agent's ability to predict the next state to introduce a new intrinsic reward that enables the agent to explore the environment.
When dealing with procedurally-generated environments, a curriculum learning mechanism can be incorporated so the episodes are ordered by an exploration score~\cite{zha2020b}, and then the agent imitates the best ones.
We also use procedurally-generated environments, but we rely on a RL approach combined with reward shaping~\cite{ng1999, jestel2021} and $\epsilon\text{-}greedy$~\cite{mnih2013} techniques to learn to navigate in them.
