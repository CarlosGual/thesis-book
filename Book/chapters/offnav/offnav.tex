\section{Offline Visual Semantic Navigation}\label{sec:offline-navigation}

In this work, we study \objnav navigation~\cite{batra2020}, a setup in which an agent is asked to navigate to a target object in an environment.
To perform this task, the agent does it using only egocentric perceptions.
Specifically, the agent receives RGB images and GPS+Compass information that provides the agent with the current position and orientation relative to the starting point.
The set of movements is discrete and consists of the following actions: \turnleft, \turnright, \moveforward, \lookup, \lookdown and \stopac.
If the agent spawns the \stopac action within 1$m$ Euclidean distance respect to the target object within a 500 steps time limit, the episode is considered successful.
In the other case, it is considered a failure.
The success rate (SR) is measured by averaging the success over all the episodes present in an evaluation set.
%We also report the Success weighted by Path Length (SPL) metric, which is the success rate weighted by the ratio between ideal and actual path length.

Since we are on an offline RL setup, we need a previously collected dataset of navigation experience.
The dataset that we chose is collected in~\cite{ramrakhya2023}.
It consists of $77k$ episodes of human navigation trajectories using the HM3D~\cite{Ramakrishnan2021HabitatMatterport3D} dataset.

We train our policies using our DD-IQL implementation on the human demonstrations.
The objetive is to find a policy with optimal parameters $\phi^*$ that maximizes the expected return from the dataset.
To do so, the IQL algorithm relies on the use of expectile regression to modify a temporal-difference (TD) loss.
This modified TD loss is able to learn an approximate Q-function from the dataset actions.
This Q-function does not explicitly represent the corresponding policy, so a separate policy extraction step is needed.
For policy extraction, we use advantage-weighted regression~\cite{peters2007, peng2019advantageweighted}:

\begin{equation}
    L_\pi(\phi)=\mathbb{E}_{(s, a) \sim \mathcal{D}}\left[\exp \left(\beta\left(Q_{\hat{\theta}}(s, a)-V_\psi(s)\right)\right) \log \pi_\phi(a|s)\right]\; ,
    \label{eq:loss}
\end{equation}

where $\beta \in [0, \infty)$ controls the trade-off between cloning the expert policy and maximizing the Q-function.
This loss can be seen as a selection of most optimal actions to clone in the dataset.
We also employ inflection weighting~\cite{wijmans2019} to modify the loss function, thereby giving more importance to those time steps where there is a change in actions.

For the policy architecture, we use a simple CNN+RNN model from\cite{ramrakhya2023}.
The difference is that we use ResNet18 for the visual encoders.
We copy the same architecture for the policy net, the Q net and the Q target net.
For the V net, we only use the visual encoder and a single linear layer, without any recurrent module.


