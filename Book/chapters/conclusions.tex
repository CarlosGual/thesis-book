\chapter{Conclusions}\label{ch:conclusions}

\lettrine{\textcolor{accent_color}{T}}{his} thesis has focused on the exploration of robotic visual navigation, specifically in the context of visual semantic navigation, and how it can be improved through the use of reinforcement learning methodologies.
The main objective has been to develop a new ROS~\cite{ros} framework that allows the integration of different algorithms and environments, as well as to explore new approaches to offline reinforcement learning and meta imitation learning for robotic visual navigation.
Concretely, the thesis has focused into studying and solving the \acrshort{objnav} problem in the real world, which consists of navigating to a specific object instance in an determined environment using visual information.
This is a challenging task since it involves a combination of several habilities, such as visual perception, semantic understanding, and navigation in complex environments.

This problem has been heavily dependent on the use of simulated environments for training and evaluation, which has limited the applicability of the developed algorithms in real-world scenarios.
However, it is not until recent years that the use of real-world data for evaluation has become more common, allowing for a better understanding of the limitations and challenges of the developed algorithms in real scenarios.
In this scenario, the thesis focuses on studying the limitations of the current approaches to robotic visual navigation in the real world, and how they can be improved through the use of new methodologies and frameworks.

Broadly, the thesis has contributed to the topics of \acrfull{vsn} and \acrshort{rl} in the context of robotic visual navigation, and how they can be improved through the use of new methodologies and frameworks.


\section{Contributions}\label{sec:contributions}
\begin{itemize}
    \item \textbf{ROS4\acrshort{vsn}:} A new framework for robotic visual navigation that allows the integration of different algorithms and environments.
    \item \textbf{Offline Reinforcement Learning for Robotic Visual Navigation:} A new approach to offline reinforcement learning for robotic visual navigation that uses a combination of imitation learning and reinforcement learning.
    \item \textbf{Meta Imitation Learning for Robotic Visual Navigation:} A new approach to meta imitation learning for robotic visual navigation that uses a combination of meta-learning and imitation learning.
\end{itemize}


\section{Discussion and Further Improvements}\label{sec:discussion-and-further-improvements}


\section{Future Research Lines}\label{sec:future-work}


\section{Scientific Contributions}\label{sec:final-remarks}

During the development of this thesis, it has been possible to contribute to topics of this thesis in several ways.
It has also been possible to contribute to other topics that, although not directly related to the thesis, have been developed in parallel.
These have come from side projects or collaborations with other research groups and have contributed to the development of this thesis.
All of them are summarized in the following subsections.

\subsection{Contributions directly related to the thesis}\label{subsec:contributions-directly-related-to-the-thesis}

\begin{itemize}
    \item \cvpub{\textbf{Gutiérrez-Alvarez C.}, Ríos-Navarro P., Flor-Rodríguez-Rabadán R., Acevedo-Rodríguez F.J., López-Sastre R.J., \textit{Visual Semantic Navigation with Real Robots}, in Applied Intelligence, 2024.}
    \item \textbf{Gutiérrez-Alvarez C.}, Acevedo-Rodríguez F.J., López-Sastre R.J., Kanezaki A., OffNav: \textit{Offline Reinforcement Learning for Visual Semantic Navigation}, in ICRA Human-aligned Reinforcement Learning for Autonomous Agents and Robots Workshop, 2024.
    \item \textbf{Gutiérrez-Alvarez C.}, Ríos-Navarro P., Flor-Rodríguez-Rabadán R., Acevedo-Rodríguez F.J., López-Sastre R.J., \textit{Evaluation of Visual Semantic Navigation Models in Real Robots}, in IROS Late Breaking Results, 2023.
    \item \textbf{Gutiérrez-Alvarez C.}, Hernández García S, Nasri N, Cuesta-Infante Alfredo, López-Sastre RJ, \textit{Towards Clear Evaluation of Robotic Visual Semantic Navigation}, in ICARA, 2023.
    \item \textit{Participation in the project \textbf{NAVIGATOR-D} (PID2023-148310OB-I00)}, funded by the Spanish Ministry of Science and Innovation.
    \item \textit{Participation in the project \textbf{AIRPLANE} (PID2019-104323RB-C31)}, funded by the Spanish Ministry of Science and Innovation.
\end{itemize}

\subsection{Side contributions}\label{subsec:side-contributions}

\begin{itemize}
    \item Flor-Rodríguez-Rabadán R., \textbf{Gutiérrez-Álvarez C.}, Acevedo-Rodríguez F.J., Lafuente-Arroyo S., López-Sastre R.J., \textit{SEMNAV: A Semantic Segmentation-Driven Approach to Visual Semantic Navigation}, in ArXiv, 2025.
    \item Blanco-Fernández E., \textbf{Gutiérrez-Alvarez C.}, Nasri N., Maldonado-Bascón S., López-Sastre R.J., \textit{Live Video Captioning}, in Multimedia Tools and Applications, 2025.
    \item \item Nasri N, \textbf{Gutiérrez-Álvarez C.}, López-Sastre RJ, Lafuente-Arroyo S., Maldonado-Bascón S. \textit{Realistic Continual Learning Approach using Pre-trained Models}, in ArXiv 2024.
    \item Lafuente-Arroyo S., Maldonado-Bascón S., Delgado-Mena D., \textbf{Gutiérrez-Alvarez C.}, Acevedo-Rodríguez F.J., \textit{Multisensory Integration for Topological Indoor Localization of Mobile Robots in Complex Symmetrical Environments}, in Expert Systems with Applications, 2023.
    \item Nasri N, López-Sastre RJ, Pacheco-da-Costa S, Fernández-Munilla I, \textbf{Gutiérrez-Álvarez C.}, Pousada-García T, Acevedo-Rodríguez FJ, Maldonado-Bascón S. \textit{Assistive Robot with an AI-Based Application for the Reinforcement of Activities of Daily Living: Technical Validation with Users Affected by Neurodevelopmental Disorders}, in Applied Sciences, 2022.
\end{itemize}