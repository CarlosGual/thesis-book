%%%%%%%%%%%%%%%%%%%%%%%%%%%%%%%%%%%%%%%%%%%%%%%%%%%%%%%%%%%%%%%%%%%%%%%%%%%
%
% Generic template for TFC/TFM/TFG/Tesis
%
% $Id: resumen.tex,v 1.9 2015/06/05 00:10:31 macias Exp $
%
% By:
%  + Javier Macías-Guarasa. 
%    Departamento de Electrónica
%    Universidad de Alcalá
%  + Roberto Barra-Chicote. 
%    Departamento de Ingeniería Electrónica
%    Universidad Politécnica de Madrid   
% 
% Based on original sources by Roberto Barra, Manuel Ocaña, Jesús Nuevo,
% Pedro Revenga, Fernando Herránz and Noelia Hernández. Thanks a lot to
% all of them, and to the many anonymous contributors found (thanks to
% google) that provided help in setting all this up.
%
% See also the additionalContributors.txt file to check the name of
% additional contributors to this work.
%
% If you think you can add pieces of relevant/useful examples,
% improvements, please contact us at (macias@depeca.uah.es)
%
% You can freely use this template and please contribute with
% comments or suggestions!!!
%
%%%%%%%%%%%%%%%%%%%%%%%%%%%%%%%%%%%%%%%%%%%%%%%%%%%%%%%%%%%%%%%%%%%%%%%%%%%

\chapter*{Resumen}
\label{cha:resumen}
\markboth{Resumen}{Resumen}

\addcontentsline{toc}{chapter}{Resumen}

Esta tesis explora la Navegación Semántica Visual (VSN por sus siglas en inglés), un desafío fundamental en robótica donde los agentes navegan en un entorno utilizando únicamente información visual sin mapas previos.
El trabajo aborda tres desafíos clave: navegar en entornos desconocidos, adaptaciones de simulación a realidad, y rendimiento en el mundo real.
La investigación abarca fundamentos teóricos, evaluaciones basadas en simulación, implementaciones en el mundo real, y nuevos enfoques algorítmicos.

La tesis comienza estableciendo un marco teórico integral para VSN, revisando métodos clásicos, enfoques de aprendizaje modular, y técnicas de aprendizaje extremo a extremo.
También se revisan métodos de exploración, de transferencia de la simulación a la realidad, de aprendizaje por refuerzo offline, y de meta-aprendizaje.
Primero se propone un modelo VSN que aprovecha codificadores CLIP combinados con redes neuronales recurrentes, entrenado usando Aprendizaje por Refuerzo (RL por sus siglas en inglés).
Para abordar el problema de las recompensas escasas inherente a tareas de navegación, la investigación evalúa técnicas de modelado de recompensas y estrategias de exploración ε-greedy.
Se desarrolla un protocolo de evaluación experimental exhaustivo usando pyRIL para dos entornos de navegación: Miniworld-Maze y Habitat con el conjunto de datos HM3D.

Una contribución significativa es el desarrollo de ROS4VSN, un marco novedoso para el Robot Operating System (ROS) que permite el despliegue y evaluación de modelos de VSN en robots reales.
Esta arquitectura modular incluye componentes para control robots, integración de cámaras, ejecución de movimientos discretos, e integración de modelos VSN.
Dos modelos VSN de vanguardia (PIRLNav y VLV) se integran en este marco y se evalúan en plataformas robóticas reales, revelando diferencias significativas de rendimiento entre entornos simulados y del mundo real.

Este trabajo también explora enfoques más allá del RL tradicional para abordar desafíos del mundo real.
OffNav (Navegación Semántica Visual Offline) implementa Q-Learning Implícito de manera distribuida descentralizada, permitiendo el entrenamiento desde conjuntos de datos fijos sin interacción con el entorno.
MetaNav (Meta Navegación Semántica Visual) combina meta-aprendizaje con aprendizaje por imitación para permitir adaptación rápida a nuevas tareas con ejemplos mínimos.
Ambos enfoques buscan cerrar la brecha entre simulación y realidad mejorando la eficiencia de muestras y la capacidad de generalización.

La investigación concluye que mientras los métodos de aprendizaje extremo a extremo son prometedores en entornos simulados, los enfoques modulares actualmente funcionan mejor en escenarios del mundo real.
Las direcciones futuras incluyen explorar diferentes arquitecturas de meta-aprendizaje, expandir más allá de la navegación de objetos hacia tareas más complejas, e incorporar modalidades sensoriales adicionales como retroalimentación auditiva y táctil.

\textbf{Palabras clave:} \myThesisKeywords.

%%% Local Variables:
%%% TeX-master: "../book"
%%% End:


