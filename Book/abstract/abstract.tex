%%%%%%%%%%%%%%%%%%%%%%%%%%%%%%%%%%%%%%%%%%%%%%%%%%%%%%%%%%%%%%%%%%%%%%%%%%%
%
% Generic template for TFC/TFM/TFG/Tesis
%
% $Id: abstract.tex,v 1.9 2015/06/05 00:10:31 macias Exp $
%
% By:
%  + Javier Macías-Guarasa. 
%    Departamento de Electrónica
%    Universidad de Alcalá
%  + Roberto Barra-Chicote. 
%    Departamento de Ingeniería Electrónica
%    Universidad Politécnica de Madrid   
% 
% Based on original sources by Roberto Barra, Manuel Ocaña, Jesús Nuevo,
% Pedro Revenga, Fernando Herránz and Noelia Hernández. Thanks a lot to
% all of them, and to the many anonymous contributors found (thanks to
% google) that provided help in setting all this up.
%
% See also the additionalContributors.txt file to check the name of
% additional contributors to this work.
%
% If you think you can add pieces of relevant/useful examples,
% improvements, please contact us at (macias@depeca.uah.es)
%
% You can freely use this template and please contribute with
% comments or suggestions!!!
%
%%%%%%%%%%%%%%%%%%%%%%%%%%%%%%%%%%%%%%%%%%%%%%%%%%%%%%%%%%%%%%%%%%%%%%%%%%%

\chapter*{Abstract}
\label{cha:abstract}

\addcontentsline{toc}{chapter}{Abstract}

This thesis explores Visual Semantic Navigation (VSN), a fundamental challenge in robotics where agents navigate in an environment using only visual information without prior maps.
The work addresses three key challenges: navigating in unknown environments, sim-to-real adaptations, and real world performance.
The research spans theoretical foundations, simulation-based evaluations, real-world implementations, and novel algorithmic approaches.

The thesis begins by establishing a comprehensive theoretical framework for VSN, reviewing classical methods, modular-learning approaches, and end-to-end learning techniques.
Exploration methods, sim-to-real transfer, offline reinforcement learning, and meta-learning methods are also reviewed.
First, a VSN model that leverages CLIP encoders combined with recurrent neural networks, trained using Reinforcement Learning (RL), is proposed.
To address the sparse reward problem inherent in navigation tasks, the research evaluates reward shaping techniques and ε-greedy exploration strategies.
A thorough experimental evaluation protocol is developed using pyRIL for two navigation environments: Miniworld-Maze and Habitat with HM3D dataset.

A significant contribution is the development of ROS4VSN, a novel Robot Operating System (ROS) framework that enables the deployment and evaluation of VSN models in real robots.
This modular architecture includes components for robot control, camera integration, discrete movement execution, and VSN model integration.
Two state-of-the-art VSN models (PIRLNav and VLV) are integrated into this framework and evaluated on real robotic platforms, revealing significant performance differences between simulated and real-world environments.

The thesis also explores approaches beyond traditional RL to address real-world challenges.
OffNav (Offline Visual Semantic Navigation) implements Implicit Q-Learning in a decentralized distributed manner, enabling training from fixed datasets without environment interaction.
MetaNav (Meta Visual Semantic Navigation) combines meta-learning with imitation learning to enable rapid adaptation to new tasks with minimal examples.
Both approaches aim to bridge the gap between simulation and reality by improving sample efficiency and generalization capabilities.

The research concludes that while end-to-end learning methods show promise in simulated environments, modular approaches currently perform better in real-world scenarios.
Future directions include exploring different meta-learning architectures, expanding beyond object navigation to more complex tasks, and incorporating additional sensory modalities like audio and tactile feedback.

\textbf{Keywords:} \myThesisKeywordsEnglish.

%%% Local Variables:
%%% TeX-master: "../book"
%%% End:


