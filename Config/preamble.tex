%%%%%%%%%%%%%%%%%%%%%%%%%%%%%%%%%%%%%%%%%%%%%%%%%%%%%%%%%%%%%%%%%%%%%%%%%%% 
% 
% Generic template for TFC/TFM/TFG/Tesis
% 
% $Id: preamble.tex,v 1.34 2017/04/06 13:56:12 macias Exp $
% 
% By:
% + Javier Macías-Guarasa. 
%   Departamento de Electrónica
%   Universidad de Alcalá
% + Roberto Barra-Chicote. 
%   Departamento de Ingeniería Electrónica
%   Universidad Politécnica de Madrid   
% 
% Based on original sources by Roberto Barra, Manuel Ocaña, Jesús Nuevo,
% Pedro Revenga, Fernando Herránz and Noelia Hernández. Thanks a lot to
% all of them, and to the many anonymous contributors found (thanks to
% google) that provided help in setting all this up.
% 
% See also the additionalContributors.txt file to check the name of
% additional contributors to this work.
% 
% If you think you can add pieces of relevant/useful examples,
% improvements, please contact us at (macias@depeca.uah.es)
% 
% You can freely use this template and please contribute with
% comments or suggestions!!!
% 
%%%%%%%%%%%%%%%%%%%%%%%%%%%%%%%%%%%%%%%%%%%%%%%%%%%%%%%%%%%%%%%%%%%%%%%%%%% 

%% FIXING PROBLEM WITH ALL PAGES PRINTED IN COLOR \documentclass[RGB,rgb,svgnames,spanish,openright]{book}
%\documentclass[spanish,openright]{book}
% \documentclass[english,openright]{book}
% \documentclass[11pt,english,twoside,openright]{book}

% \usepackage[a4,cam,center]{crop}
% \crop[font=\upshape\mdseries\small\textsf]

\synctex=1

\usepackage{chapters/ros4vsn/symbols}

% To generate a proper PDF/A document
%\usepackage[a-1b]{pdfx}

% To allow changing the default alignment of an image
\usepackage[export]{adjustbox}

% To allow simple notes to be used in the review process (see defined
% commands at the end of this file)
\usepackage{todonotes}

% To allow ues of epigraphs
\usepackage{epigraph}
\usepackage{lettrine}

% ifthen to allow using language dependent settings
\usepackage{ifthen}

%% JMG: FIXING PROBLEM WITH ALL PAGES PRINTED IN COLOR
% This should not be touched, as it should work as it is know.
\newcommand{\colorspaceused}{rgb}

%The next section seems to be useless, but it's still pending to try further
\ifthenelse{\equal{\colorspaceused}{rgb}}
{
  \PassOptionsToPackage{rgb}{xcolor}% NB: put this *before* \usepackage{pst-all}
}
{
  \PassOptionsToPackage{cmyk}{xcolor}% NB: put this *before* \usepackage{pst-all}
}

\usepackage{iftex}
%\usepackage[latin1]{inputenc} % Para poder escribir con acentos y ñ. en
                              % latin1
\ifPDFTeX
  \usepackage[utf8]{inputenc} % Para poder escribir con acentos y ñ.
  \usepackage[T1]{fontenc}      % Para que haga bien la ``hyphenation''. No
\fi                                % usar si no es necesario, porque ralentiza muchisimo la compilación.
\usepackage{ae}               % Para que todas las fuentes sean Type1, y ninguna Type3.
\usepackage{lmodern}          % This generates a pdf with searchable
                              % accented characters!!!!!!!!!!!!!!!!!!!!!!!!!!!!!!!!!!!!!!!


\usepackage{wrapfig}
\usepackage{lipsum}

% Use this if you want to include pdf files in the final document
\usepackage[final]{pdfpages}

% Use this if you want to delete headers and footers in empty pages
\usepackage{emptypage}

% \usepackage[nottoc]{tocbibind}
\usepackage{tocbibind}

\usepackage{listings}
\usepackage{longtable}
\usepackage{afterpage}

\usepackage{xspace}
\usepackage{verbatim}
\usepackage{moreverb}
\usepackage{multicol}
\usepackage{amsmath}
\usepackage{eurosym}
%\usepackage{subfig} % subfigure is obsolete... 
\usepackage{multirow}
\usepackage{fancyhdr}
\usepackage{makeidx}
\usepackage{rotating}
\usepackage{supertabular}
\usepackage{hhline}
\usepackage{array}



%% FIXING PROBLEM WITH ALL PAGES PRINTED IN COLOR
\usepackage{xcolor}
% \usepackage[RGB,rgb]{xcolor}
% \usepackage{color}
% Pantone 160
% \definecolor{headingPortadaTFM}{RGB}{158,84,10}
% Pantone 160C (this is supposed to be the correct one, but it looks horrible in screen)
% \definecolor{headingPortadaTFM}{RGB}{161,86,28}
% Gold in RGB
% \definecolor{textoHeadingPortadaTFM}{RGB}{215,215,0}
% Captured colors in screen (this looks pst on screen)

% \ifthenelse{\equal{\colorspaceused}{rgb}}
% {
%   \definecolor{headingPortadaTFM}{RGB}{152,118,52}
%   \definecolor{textoHeadingPortadaTFM}{RGB}{208,205,102}
% }
% {
%   % These definitions are for cmyk colorspace
%   \definecolor{headingPortadaTFM}{cmyk}{0.0254,0,0.559,0.537}
%   \definecolor{textoHeadingPortadaTFM}{cmyk}{0,0.0144,0.51,0.184}
% }

\definecolor{pantone293}{RGB}{35,91,168}

\definecolor{headingPortadaTFG}{RGB}{152,118,52}
\definecolor{headingPortadaTFM}{RGB}{0,90,170}
\definecolor{textoHeadingPortadaTFM}{RGB}{208,205,102}
\definecolor{textoHeadingPortadaTFG}{RGB}{208,205,102}

\definecolor{gray97}{gray}{.97}
\definecolor{gray75}{gray}{.75}
\definecolor{gray45}{gray}{.45}




% To draw rectagles in tfm cover
\usepackage{tikz}


% \usepackage[authoryear]{natbib}
% \makeatletter
% \let\NAT@parse\undefined
% \makeatother
% \usepackage{natbib}

\usepackage{geometry}
\geometry{verbose,a4paper,tmargin=2.5cm,bmargin=2.5cm,lmargin=2.5cm,rmargin=2.5cm}
% \geometry{paperwidth=210mm,paperheight=297mm}

%\usepackage[hang, flushmargin]{footmisc}   

\usepackage{hyperref}
\usepackage{hyperxmp}
\hypersetup{
%% ps2pdf,                %%% hyper-references for ps2pdf
bookmarks=true,%                   %%% generate bookmarks ...
bookmarksnumbered=true,            %%% ... with numbers
hypertexnames=false,               %%% needed for correct links to
%%% figures!!!
% hypertexnames=true,               %%% needed for correct links on pagebackrefs!!!
breaklinks=true,                   %%% breaks lines, but links are very small
% pagebackref=true,
% linktocpage=true,                 %%% enlace en el numero de página.
linktoc=all,
colorlinks=true,
linkcolor=blue,    
citecolor=green,
urlcolor=blue,                     %%% texto  con color (further
%%% modified in myconfig.tex)
% linkbordercolor={0 0 1},           %%% blue frames around links
pdfborder={0 0 112.0},              %%% border-width of frames 
hyperfootnotes=false
}                        %%% will be multiplied with 0.009 by ps2pdf


% \usepackage[all]{hypcap}
\usepackage[center]{caption}
\usepackage{subcaption}


% Para numerar las \subsubsection
\setcounter{secnumdepth}{5}
% para hacer que las \subsubsection aparezcan en el indice
\setcounter{tocdepth}{5}
% \setcounter{lofdepth}{2}
\setcounter{table}{1}
\setcounter{figure}{1}
\setcounter{secnumdepth}{4}


\setlength{\parskip}{1ex plus 0.5ex minus 0.2ex}


\usepackage{multirow}

\usepackage{setspace}
% \renewcommand{\baselinestretch}{10}
\newcommand{\mycaptiontable}[1]{
  \begin{spacing}{0.6}
    % \vspace{0.5cm}
    \begin{quote}
      % \begin{center}
      {{Table} \thechapter.\arabic{table}: #1}
      % \end{center}
    \end{quote}
    % \vspace{1cm}
  \end{spacing}
  \stepcounter{table}
}

\newcommand{\mycaptionfigure}[1]{
  % \vspace{0.5cm}
  \begin{spacing}{0.6}
    \begin{quote}
      % \begin{center}
      {{Figure} \thechapter.\arabic{figure}: #1}
      % \end{center}
    \end{quote}
    % \vspace{1cm}
  \end{spacing}
  \stepcounter{figure}
}

\usepackage{amsmath}

\usepackage{courier}

% ***************************************************************************
% ***************************************************************************
% ***************************************************************************
\usepackage{multirow}
\usepackage{rotating}
\usepackage{setspace, amssymb, amsmath, epsfig, multirow, colortbl, tabularx}%
% For acronym package:
% If footnote is specified, text will be included in a footnote
% If printonlyused is specified, only used acronyms will be included
% I use the acronym sty under the sty directory as I needed the newest version
% \usepackage[footnote,printonlyused,withpage]{acronym} 
% \usepackage[printonlyused]{sty/acronym}

% glossaries is better than the acronym package 
\usepackage[automake,acronym,shortcuts,nomain,hyperfirst=false]{glossaries}
% If you want to PERMANENTLY DISABLE HYPERLINKS, uncomment the following
% line
% \glsdisablehyper
% In future versiones (not as for ubuntu 12.04) You can also selectively
% disable hyperlinks for given glossaries, using:
% \usepackage[acronym,shortcuts,nomain,nohypertypes={acronyms,symbols}]{glossaries}
% Or (for newwer versions also), you can even use
% \GlsDeclareNoHyperList{acronyms,symbols}
% You can also disable hyperlinks in the acronym use, like in \ac*{symbol}


\newcommand{\clearemptydoublepage}{\newpage{\pagestyle{empty}\cleardoublepage}}

\pagestyle{fancy}

\providecommand\phantomsection{}
\onehalfspacing
\sloppy  %better line breaks

\renewcommand{\chaptermark}[1]{\markboth{\chaptername\ \thechapter.\ #1}{}}
\renewcommand{\sectionmark}[1]{\markright{\thesection\ #1}{}}

%%%%%%%%%%%%%%%%%%%%%%%%%%%%%%%%%%%%%%%%%%%%%%%%%%%%%%%%%%%%%%%%%%%%%%%%%%% 
% BEGIN Fancy headers stuff
\fancyhf{}

\fancyhead[LE,RO]{\bfseries\thepage}
\fancyhead[LO]{\bfseries\rightmark}
\fancyhead[RE]{\bfseries\leftmark}

\makeatletter
\renewcommand{\chaptermark}[1]{\markboth{\@chapapp \ \thechapter . \ #1}{}}
\renewcommand{\sectionmark}[1]{\markright{\thesection \ \ #1}}
\makeatother

\renewcommand{\headrulewidth}{0.5pt}
\renewcommand{\footrulewidth}{0pt}
\addtolength{\headheight}{3.5pt}
\fancypagestyle{plain}{\fancyhead{}\renewcommand{\headrulewidth}{0pt}}
\fancypagestyle{myplain}
{
  \fancyhf{}
  \renewcommand\headrulewidth{0pt}
  \renewcommand\footrulewidth{0pt}
  \fancyfoot[C]{\thepage}
}
% END Fancy headers stuff
%%%%%%%%%%%%%%%%%%%%%%%%%%%%%%%%%%%%%%%%%%%%%%%%%%%%%%%%%%%%%%%%%%%%%%%%%%% 

%%%%%%%%%%%%%%%%%%%%%%%%%%%%%%%%%%%%%%%%%%%%%%%%%%%%%%%%%%%%%%%%%%%%%%%%%%% 
% BEGIN Set nice chapter titles

% BEGIN Example 0 from http://texblog.org/2012/07/03/fancy-latex-chapter-styles/
% \usepackage[explicit]{titlesec}
% \usepackage{blindtext}
% \definecolor{gray75}{gray}{0.75}
% \newcommand{\hsp}{\hspace{20pt}}
% \titleformat{\chapter}[hang]{\Huge\bfseries}{\chaptername~\thechapter\hsp\textcolor{gray75}{|}\hsp}{0pt}{\Huge\bfseries}
% END Example 0 from http://texblog.org/2012/07/03/fancy-latex-chapter-styles/

% BEGIN Example 1 from http://texblog.org/2012/07/03/fancy-latex-chapter-styles/
% \usepackage{titlesec}
% \usepackage{blindtext}
% \definecolor{gray75}{gray}{0.75}
% \newcommand{\hsp}{\hspace{20pt}}
% \titleformat{\chapter}[hang]{\Huge\bfseries}{\chaptername~\thechapter\hsp\textcolor{gray75}{|}\hsp}{0pt}{\Huge\bfseries}
% END Example 1 from http://texblog.org/2012/07/03/fancy-latex-chapter-styles/

% BEGIN Example 2 from http://texblog.org/2012/07/03/fancy-latex-chapter-styles/
% Options: Sonny, Lenny, Glenn, Conny, Rejne, Bjarne, Bjornstrup
% \usepackage[Sonny]{fncychap}
% \usepackage[Lenny]{fncychap} % ugly
% \usepackage[Glenn]{fncychap}
% \usepackage[Conny]{fncychap} % ugly
% \usepackage[Rejne]{fncychap}
% \usepackage[Bjarne]{fncychap} % Doesn't work in Spanish
% \usepackage[Bjornstrup]{fncychap}
% END   Example 2 from http://texblog.org/2012/07/03/fancy-latex-chapter-styles/

% BEGIN Example 3 from http://texblog.org/2012/07/03/fancy-latex-chapter-styles/
% This is a nice colored example
% \usepackage{kpfonts}
% \usepackage[explicit]{titlesec}
% \newcommand*\chapterlabel{}
% \titleformat{\chapter}
% {\gdef\chapterlabel{}
% \normalfont\sffamily\Huge\bfseries\scshape}
% {\gdef\chapterlabel{\thechapter\ }}{0pt}
% {\begin{tikzpicture}[remember picture,overlay]
%   \node[yshift=-3cm] at (current page.north west)
%   {\begin{tikzpicture}[remember picture, overlay]
%     \draw[fill=LightSkyBlue] (0,0) rectangle
%     (\paperwidth,3cm);
%     \node[anchor=east,xshift=.9\paperwidth,rectangle,
%     rounded corners=20pt,inner sep=11pt,
%     fill=MidnightBlue]
%     {\color{white}\chapterlabel#1};
%   \end{tikzpicture}
% };
% \end{tikzpicture}
% }
%   \titlespacing*{\chapter}{0pt}{50pt}{-60pt}
%   END   Example 3 from http://texblog.org/2012/07/03/fancy-latex-chapter-styles/

%   BEGIN Example 4 from http://texblog.org/2012/07/03/fancy-latex-chapter-styles/
%   END   Example 4 from http://texblog.org/2012/07/03/fancy-latex-chapter-styles/


%   END Set nice chapter titles
%%%%%%%%%%%%%%%%%%%%%%%%%%%%%%%%%%%%%%%%%%%%%%%%%%%%%%%%%%%%%%%%%%%%%%%%%%%   

%%%%%%%%%%%%%%%%%%%%%%%%%%%%%%%%%%%%%%%%%%%%%%%%%%%%%%%%%%%%%%%%%%%%%%%%%%%   
%   This is to set background images (in our case to set background image
%   in TFMs front and back pages)
%   If you want to set this background, use \BgThispage in the
%   corresponding pages
%\usepackage[pages=some]{sty/background}
\usepackage[pages=some]{background}

% Note that we also set the opacity in the first page of the tfg due to a bug,
% so it you modify it here remember to modify the value in Book/cover/portada-tfm-uah.tex
% https://tex.stackexchange.com/questions/649514/how-to-produce-a-transparent-image-with-xelatex/649518#649518
% https://tex.stackexchange.com/questions/640574/using-a-tikzpicture-disables-opacity-for-first-bgthispage-how-can-i-fix-this
\ifthenelse{\equal{\colorspaceused}{rgb}}
{
  \backgroundsetup{ scale=1, angle=0, opacity=.1, color=pink,
    contents={\includegraphics[width=.7\paperwidth]{logos/logoEPS-UAH.jpg}}, vshift=-50pt,  hshift=0pt }
}
{
  \backgroundsetup{ scale=1, angle=0, opacity=.1, color=pink,
    contents={\includegraphics[width=.7\paperwidth]{logos/logoEPS-UAH-cmyk.jpg}}, vshift=-50pt,  hshift=0pt }
}


% This is to allow do a clearpage and let the next one to be placed in
% even pages (to set a backpage for example)
\makeatletter
\newcommand*{\cleartoleftpage}{%
  \clearpage
  \if@twoside
  \ifodd\c@page
  \hbox{}\newpage
  \if@twocolumn
  \hbox{}\newpage
  \fi
  \fi
  \fi
}
\makeatother

% Let's define some styles for source code listings:
% 
% minimizar fragmentado de listados (from
% http://www.rafalinux.com/?p=599), pero no me funciona:
% \lstnewenvironment{codelisting}[1][]
% {\lstset{#1}\pagebreak[0]}{\pagebreak[0]}
% 
% This was using the float package
\usepackage{float}
\floatstyle{plaintop} % optionally change the style of the new float
\newfloat{codefloat}{H}{cod}[chapter]

% Support utf-8 in listings. 
% The way the inputenc package works with non-ASCII UTF-8-encoded characters (by
% making the first byte active and then reading the following ones as arguments)
% is fundamentally incompatible with the way the listing package works, which
% reads each byte individually and expects it to be an individual character.
% See https://tex.stackexchange.com/questions/24528/having-problems-with-listings-and-utf-8-can-it-be-fixed
\lstset{
    inputencoding = utf8,  % Input encoding
    extendedchars = true,  % Extended ASCII
    literate      =        % Support additional characters
      {á}{{\'a}}1  {é}{{\'e}}1  {í}{{\'i}}1 {ó}{{\'o}}1  {ú}{{\'u}}1
      {Á}{{\'A}}1  {É}{{\'E}}1  {Í}{{\'I}}1 {Ó}{{\'O}}1  {Ú}{{\'U}}1
      {à}{{\`a}}1  {è}{{\`e}}1  {ì}{{\`i}}1 {ò}{{\`o}}1  {ù}{{\`u}}1
      {À}{{\`A}}1  {È}{{\'E}}1  {Ì}{{\`I}}1 {Ò}{{\`O}}1  {Ù}{{\`U}}1
      {ä}{{\"a}}1  {ë}{{\"e}}1  {ï}{{\"i}}1 {ö}{{\"o}}1  {ü}{{\"u}}1
      {Ä}{{\"A}}1  {Ë}{{\"E}}1  {Ï}{{\"I}}1 {Ö}{{\"O}}1  {Ü}{{\"U}}1
      {â}{{\^a}}1  {ê}{{\^e}}1  {î}{{\^i}}1 {ô}{{\^o}}1  {û}{{\^u}}1
      {Â}{{\^A}}1  {Ê}{{\^E}}1  {Î}{{\^I}}1 {Ô}{{\^O}}1  {Û}{{\^U}}1
      {œ}{{\oe}}1  {Œ}{{\OE}}1  {æ}{{\ae}}1 {Æ}{{\AE}}1  {ß}{{\ss}}1
      {ç}{{\c c}}1 {Ç}{{\c C}}1 {ø}{{\o}}1  {Ø}{{\O}}1   {å}{{\r a}}1
      {Å}{{\r A}}1 {ã}{{\~a}}1  {õ}{{\~o}}1 {Ã}{{\~A}}1  {Õ}{{\~O}}1
      {ñ}{{\~n}}1  {Ñ}{{\~N}}1  {¿}{{?`}}1  {¡}{{!`}}1
      {°}{{\textdegree}}1 {º}{{\textordmasculine}}1 {ª}{{\textordfeminine}}1
      % ¿ and ¡ are not correctly displayed if inconsolata font is used
      % together with the lstlisting environment. Consider typing code in
      % external files and using \lstinputlisting to display them instead.      
  }

\lstdefinestyle{console}
{
  basicstyle=\scriptsize\bf\ttfamily,
  backgroundcolor=\color{gray75},
}

\lstdefinestyle{Cbluebox}
{
  language=C,
  frame=shadowbox, 
  rulesepcolor=\color{blue}
}

\lstdefinestyle{Cnice}
{
  language=C,
  frame=Ltb,
  framerule=0pt,
  tabsize=2,
  aboveskip=0.5cm,
  framextopmargin=3pt,
  framexbottommargin=3pt,
  framexleftmargin=0.4cm,
  framesep=0pt,
  rulesep=.4pt,
  backgroundcolor=\color{gray97},
  rulesepcolor=\color{black},
  % 
  stringstyle=\ttfamily,
  showstringspaces = false,
  % basicstyle=\small\ttfamily,
  basicstyle=\footnotesize\ttfamily,
  commentstyle=\color{gray45},
  keywordstyle=\bfseries,
  % 
  numbers=left,
  numbersep=15pt,
  numberstyle=\tiny,
  numberfirstline = false,
  breaklines=true,
}	

\lstdefinestyle{CppExample}
{
  language=C++,
  frame=trbl,
  tabsize=2,
  commentstyle=\textit,
  stringstyle=\ttfamily, 
  basicstyle=\small,
}	

% This one from http://en.wikibooks.org/wiki/LaTeX/Source_Code_Listings
\lstdefinestyle{Ccolor}
{
  belowcaptionskip=1\baselineskip,
  breaklines=true,
  frame=L,
  xleftmargin=\parindent,
  language=C,
  showstringspaces=false,
  basicstyle=\footnotesize\ttfamily,
  keywordstyle=\bfseries\color{green!40!black},
  commentstyle=\itshape\color{purple!40!black},
  identifierstyle=\color{blue},
  stringstyle=\color{orange},
}

% From http://tex.stackexchange.com/questions/46953/unix-command-highlighting-latex
\lstdefinestyle{BashInputStyle}{
  language=bash,
  basicstyle=\small\sffamily,
  numbers=left,
  numberstyle=\tiny,
  numbersep=3pt,
  frame=tb, 
  showspaces=false, 
  showtabs=false,
  showstringspaces=false,
  columns=fullflexible,
  backgroundcolor=\color{gray97},
  % backgroundcolor=\color{yellow!20},
  linewidth=0.9\linewidth,
  xleftmargin=0.05\linewidth
}


% To set side-captions in figures
\usepackage{sidecap}

%%%%%%%%%%%%%%%%%%%%%%%%%%%%%%%%%%%%%%%%%%%%%%%%%%%%%%%%%%%%%%%%%%%%%%%%%%% 
% This comes from TeXiS, thanks to its authors, available at
% http://gaia.fdi.ucm.es/projects/texis 
\def\texis{\TeX \raise.15em\hbox{\textsc{i}}S}
%%%%%%%%%%%%%%%%%%%%%%%%%%%%%%%%%%%%%%%%%%%%%%%%%%%%%%%%%%%%%%%%%%%%%% 
% Comando:
% 
% \begin{FraseCelebre}
%   \begin{Frase}
%     Y así, del mucho leer y del poco dormir...
%   \end{Frase}
%   \begin{Fuente}
%     Don Quijote de la Mancha
%     
%     Miguel de Cervantes
%   \end{Fuente}
%   \begin{FraseCelebre}
%     
%     Resultado:
%     
%     Añade la frase célebre del principio de un capítulo.
%%%%%%%%%%%%%%%%%%%%%%%%%%%%%%%%%%%%%%%%%%%%%%%%%%%%%%%%%%%%%%%%%%%%%%     
\newenvironment{FraseCelebre}% Definición del entorno de FraseCelebre
{\begin{list}{}{%
      \setlength{\leftmargin}{0.5\textwidth}% Desplazamos el inicio de
      % los párrafos a la derecha la mitad
      % de la anchura de la línea de texto.
      % Puede que quieras cambiar esto
      % por otra cantidad como '5cm'.
      \setlength{\parsep}{0cm}% La separación entre párrafos de la
      % frase o de la fuente es normal, sin
      % espacio extra.
      \addtolength{\topsep}{0.5cm}% Aumentamos un poco la separación
      % entre la parte de la fase célebre
      % y los párrafos de alrededor
    }
  }
  {\unskip \end{list}}

\newenvironment{Frase}%
{\item \begin{flushright}\small\em}%
  {\end{flushright}}

\newenvironment{Fuente}%
{\item \begin{flushright}\small}%
  {\end{flushright}}


% To put paragraphs at page bottom
\newenvironment{bottomparagraph}{\par\vspace*{\fill}}{\clearpage}
% \newenvironment{bottomparagraph}{\par\vspace*{\fill}}{\clearemptydoublepage}

% Add algorithms april 2014
\usepackage[vlined,algochapter]{algorithm2e}
% Make this compatible with older/newer versions of the package
\providecommand{\DontPrintSemicolon}{\dontprintsemicolon}
\providecommand{\SetAlgoLined}{\SetLine}



% Add support for fonts at arbitrary sizes september 2014, for TFG's cover
\usepackage{fix-cm}

\usepackage{graphicx}                                                                      

% This is to avoid producing an hyperlink for starred documents. ONLY
% WORKS FOR THE ACRONYM PACKAGE, NOT USED HERE ANYMORE
% \makeatletter
% \AtBeginDocument{%
%   \renewcommand*\AC@hyperlink{%
%     \ifAC@starred
%       \expandafter\@secondoftwo
%     \else
%       \expandafter\hyperlink
%     \fi
%   }%
% }
% \makeatother

% This should be relative to the book.tex path, do not touch!!!!!!!!!!!
\newcommand{\myreferencespath}{}


%\providecommand{\DIFadd}[1]{{\protect\color{blue}#1}} %DIF PREAMBLE
%\providecommand{\DIFdel}[1]{{\protect\color{red}\protect\scriptsize{#1}}}

% As fancy underlining does not seem to compile with pdflatex, remove underline
%\providecommand{\DIFadd}[1]{{\protect\color{blue}{\protect\uwave{#1}}}}
%\providecommand{\DIFadd}[1]{{\protect\color{blue}\textbf{#1}}}
\providecommand{\DIFdel}[1]{{\protect\color{red}\sout{#1}}}                     


%%%%%%%%%%%%%%%%%%%%%%%%%%%%%%%%%%%%%%%%%%%%%%%%%%%%%%%%%%%%%%%%%%%%%%%%%%%
% 
\usepackage{ifpdf}
\ifpdf
  \DeclareGraphicsExtensions{.pdf,.png,.jpg}
\else
  \DeclareGraphicsExtensions{.eps}
\fi

\DeclareGraphicsExtensions{.pdf,.png,.jpg}


%%%%%%%%%%%%%%%%%%%%%%%%%%%%%%%%%%%%%%%%%%%%%%%%%%%%%%%%%%%%%%%%%%%%%%%%%%%
% Para control de viudas y huérfanas
\clubpenalty=10000
\widowpenalty=10000

%%%%%%%%%%%%%%%%%%%%%%%%%%%%%%%%%%%%%%%%%%%%%%%%%%%%%%%%%%%%%%%%%%%%%%%%%%%
% To allow checking for initial letter of string being a given one
\usepackage{xstring}

%%%%%%%%%%%%%%%%%%%%%%%%%%%%%%%%%%%%%%%%%%%%%%%%%%%%%%%%%%%%%%%%%%%%%%%%%%%
% To allow bold + tt (from https://tex.stackexchange.com/questions/215482/how-do-i-get-texttt-with-bold-face-in-latex)
\usepackage{bold-extra}

%%%%%%%%%%%%%%%%%%%%%%%%%%%%%%%%%%%%%%%%%%%%%%%%%%%%%%%%%%%%%%%%%%%%%%%%%%%
% As requested by Carlos Cruz on August 2020 To make "Appendix X
% Appendix title" in TOC instead of simply "X Appendix title" (from
% https://tex.stackexchange.com/questions/44858/adding-the-word-appendix-to-table-of-contents-in-latex/44971
% and
% https://tex.stackexchange.com/questions/58848/ap%C3%A9ndices-appendix-spanish-accent):
\usepackage[titletoc]{appendix}
\usepackage{etoolbox}
\makeatletter
\appto{\appendices}{\def\Hy@chapapp{Appendix}}
\makeatother


%%%%%%%%%%%%%%%%%%%%%%%%%%%%%%%%%%%%%%%%%%%%%%%%%%%
% Bibliography backend control. It is recommended  that we use biblatex, as it
% supports more keys (for example, when we cite a website we can specify the
% visited date, in the .bib file). It also support multiple files more easily
% and more bibliography styles

%\newcommand{\bibliosystem}{bibtex} % Valid options are biblatex or bibtex
\newcommand{\bibliosystem}{biblatex} % Valid options are biblatex or bibtex

\ifthenelse{\equal{\bibliosystem}{biblatex}}
{
  % Suggestion by Miguel Cubero (2023)
  % When using babel or polyglossia with biblatex, loading csquotes is
  % recommended to ensure that quoted texts are typeset according to the
  % rules of your main language.

  \usepackage{csquotes}

  % Use biblatex instead of bibtex
  \usepackage[backend=biber,style=ieee,maxnames=99]{biblatex}
  % This is a dirty hack, but should work... The reason to do so is to avoid
  % the need of editing this file by the user (see Book/biblio files for more
  % details)
  %% Here define as many bibfiles as needed
%%
%% It is compulsory that they are named as \mybibfileOne
%% \mybibfileTwo, \mybibfileThree, ... \mybibfileTen
%%
%% If you need more than ten, you will have to edit
%% Config/preamble.tex and Book/biblio/bibliography.tex
%% to support this adition
%%
%% The file names may change at your will, but they must
%% be in the Book/biblio directory

\newcommand{\mybibfileOne}{biblio/library.bib}
\newcommand{\mybibfileTwo}{biblio/library_icara.bib}
\newcommand{\mybibfileThree}{biblio/library_harl.bib}
%% \newcommand{\mybibfileFour}{biblio/audiotracking.bib}
%% \newcommand{\mybibfileFive}{biblio/audiovisualtracking.bib}
%% \newcommand{\mybibfileSix}{biblio/backgroungsubstraction.bib}
%% \newcommand{\mybibfileSeven}{biblio/databases.bib}
%% \newcommand{\mybibfileEight}{biblio/evalmetrics.bib}
%% \newcommand{\mybibfileNine}{biblio/facedetect.bib}
%% \newcommand{\mybibfileTen}{biblio/facedetectADABOOST.bib}
%% \newcommand{\mybibfileEleven}{biblio/facedetectmultiview.bib}
%% \newcommand{\mybibfileTwelve}{biblio/facedetectprob2d.bib}
%% \newcommand{\mybibfileThirteen}{biblio/others.bib}
%% \newcommand{\mybibfileFourteen}{biblio/skindetect.bib}
%% \newcommand{\mybibfileFifteen}{biblio/tracking.bib}
%% \newcommand{\mybibfileSixteen}{biblio/videotracking.bib}
%% \newcommand{\mybibfileSeventeen}{biblio/voiceActivityDetection.bib}
%% \newcommand{\mybibfileEighteen}{biblio/headposeextraction.bib}
%% \newcommand{\mybibfileNineteen}{biblio/AudioVisualSpeakerTracking.bib}
%% \newcommand{\mybibfileTwenty}{biblio/BibliogPFVJ.bib}
%% \newcommand{\mybibfileTwentyone}{biblio/tools.bib}
%% \newcommand{\mybibfileTwentytwo}{biblio/infrared.bib}
%% \newcommand{\mybibfileTwentythree}{}
%% \newcommand{\mybibfileTwentyfour}{}
%% \newcommand{\mybibfileTwentyfive}{}



  \ifdef{\mybibfileOne}
  {
    \addbibresource{../Book/biblio/library.bib}
  }
  {
    \errorYOUmustDEFINEatLEASTmybibfileOneInbibliofilesDOTtex
  }
  \ifdef{\mybibfileTwo}
  {
    \addbibresource{\myreferencespath\mybibfileEight}
  }
  {
  }
  \ifdef{\mybibfileThree}
  {
    \addbibresource{\myreferencespath\mybibfileEight}
  }
  {
  }
  \ifdef{\mybibfileFour}
  {
    \addbibresource{\myreferencespath\mybibfileFour}
  }
  {
  }
  \ifdef{\mybibfileFive}
  {
  \addbibresource{\myreferencespath\mybibfileFive}
  }
  {
  }
  \ifdef{\mybibfileSix}
  {
    \addbibresource{\myreferencespath\mybibfileSix}
  }
  {
  }

  \ifdef{\mybibfileSeven}
  {
    \addbibresource{\myreferencespath\mybibfileSeven}
  }
  {
  }
  \ifdef{\mybibfileEight}
  {
    \addbibresource{\myreferencespath\mybibfileEight}
  }
  {
  }

  \ifdef{\mybibfileNine}
  {
    \addbibresource{\myreferencespath\mybibfileNine}
  }
  {
  }

  \ifdef{\mybibfileTen}
  {
    \addbibresource{\myreferencespath\mybibfileTen}
  }
  {
  }

  \ifdef{\mybibfileEleven}
  {
    \addbibresource{\myreferencespath\mybibfileEleven}
  }
  {
  }

  \ifdef{\mybibfileTwelve}
  {
    \addbibresource{\myreferencespath\mybibfileTwelve}
  }
  {
  }

  \ifdef{\mybibfileThirteen}
  {
    \addbibresource{\myreferencespath\mybibfileThirteen}
  }
  {
  }

  \ifdef{\mybibfileFourteen}
  {
    \addbibresource{\myreferencespath\mybibfileFourteen}
  }
  {
  }

  \ifdef{\mybibfileFifteen}
  {
    \addbibresource{\myreferencespath\mybibfileFifteen}
  }
  {
  }

  \ifdef{\mybibfileSixteen}
  {
    \addbibresource{\myreferencespath\mybibfileSixteen}
  }
  {
  }

  \ifdef{\mybibfileSeventeen}
  {
    \addbibresource{\myreferencespath\mybibfileSeventeen}
  }
  {
  }

  \ifdef{\mybibfileEighteen}
  {
    \addbibresource{\myreferencespath\mybibfileEighteen}
  }
  {
  }

  \ifdef{\mybibfileNineteen}
  {
    \addbibresource{\myreferencespath\mybibfileNineteen}
  }
  {
  }

  \ifdef{\mybibfileTwenty}
  {
    \addbibresource{\myreferencespath\mybibfileTwenty}
  }
  {
  }

  \ifdef{\mybibfileTwentyone}
  {
    \addbibresource{\myreferencespath\mybibfileTwentyone}
  }
  {
  }

  \ifdef{\mybibfileTwentytwo}
  {
    \addbibresource{\myreferencespath\mybibfileTwentytwo}
  }
  {
  }

  \ifdef{\mybibfileTwentythree}
  {
    \addbibresource{\myreferencespath\mybibfileTwentythree}
  }
  {
  }

  \ifdef{\mybibfileTwentyfour}
  {
    \addbibresource{\myreferencespath\mybibfileTwentyfour}
  }
  {
  }

  \ifdef{\mybibfileTwentyfive}
  {
    \addbibresource{\myreferencespath\mybibfileTwentyfive}
  }
  {
  }

}
{
  % Use bibtex
  \usepackage[noadjust]{cite}      % Written by Donald Arseneau
  % V1.6 and later of IEEEtran pre-defines the format
  % of the cite.sty package \cite{} output to follow
  % that of IEEE. Loading the cite package will
  % result in citation numbers being automatically
  % sorted and properly "ranged". i.e.,
  % [1], [9], [2], [7], [5], [6]
  % (without using cite.sty)
  % will become:
  % [1], [2], [5]--[7], [9] (using cite.sty)
  % cite.sty's \cite will automatically add leading
  % space, if needed. Use cite.sty's noadjust option
  % (cite.sty V3.8 and later) if you want to turn this
  % off. cite.sty is already installed on most LaTeX
  % systems. The latest version can be obtained at:
  % http://www.ctan.org/tex-archive/macros/latex/contrib/supported/cite/
}


%\input{../Config/json-lang-lstlisting.tex}

% I discovered lot of people used \input{Book/chapters/whatever.tex}, but the Book/ part is
% not needed and breaks compilation if you try to compile it under GNU/Linux. This solves it
% You can add additional directories if required
\makeatletter
\def\input@path{
  {../},  % Path for general \input \include
}
\makeatother


% From https://tex.stackexchange.com/questions/50830/do-i-have-to-care-about-bad-boxes/50850#50850
% To hide warning messages about slightly overfilled paragraphs
% This should not be here but I want to avoid adding extra files to do tiny things...
\hfuzz=2pt
\vfuzz=2pt

% Some TFG/TFM regulations state that double spacing should be used In my
% opinion it is obsolete and ugly, but if you want to do so, add the following
% command here:
% \renewcommand{\baselinestretch}{2}

% Some TFG/TFM regulations state that Arial font should be used. In my
% opinion the LaTeX standard fornt is better, but if you want to do so, use Helvetica instead (Arial is not free) adding the following commands here:
%\usepackage{helvet}
%\renewcommand{\familydefault}{\sfdefault}

% I would like to use montserrat font so I include it here:
\usepackage{montserrat}
\renewcommand{\familydefault}{\sfdefault}

%%% Local Variables:
%%% TeX-master: "../book"
%%% End:


